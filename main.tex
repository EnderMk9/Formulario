\documentclass[10pt,landscape,letterpaper]{article}
\usepackage{amssymb,amsmath,amsthm,amsfonts}
\usepackage{multicol,multirow}
\usepackage{spverbatim}
\usepackage{graphicx}
\usepackage{ifthen}
\usepackage[landscape]{geometry}
\usepackage[colorlinks=true,urlcolor=olgreen]{hyperref}
\usepackage{booktabs}
\usepackage{fontspec}
\setmainfont[Ligatures=TeX]{TeX Gyre Pagella}
\setsansfont{Fira Sans}
\setmonofont{Inconsolata}
\usepackage{unicode-math}
\setmathfont{TeX Gyre Pagella Math}
\usepackage{microtype}
% dirty fix for microtype
\makeatletter
\def\MT@is@composite#1#2\relax{%
  \ifx\\#2\\\else
    \expandafter\def\expandafter\MT@char\expandafter{\csname\expandafter
                    \string\csname\MT@encoding\endcsname
                    \MT@detokenize@n{#1}-\MT@detokenize@n{#2}\endcsname}%
    % 3 lines added:
    \ifx\UnicodeEncodingName\@undefined\else
      \expandafter\expandafter\expandafter\MT@is@uni@comp\MT@char\iffontchar\else\fi\relax
    \fi
    \expandafter\expandafter\expandafter\MT@is@letter\MT@char\relax\relax
    \ifnum\MT@char@ < \z@
      \ifMT@xunicode
        \edef\MT@char{\MT@exp@two@c\MT@strip@prefix\meaning\MT@char>\relax}%
          \expandafter\MT@exp@two@c\expandafter\MT@is@charx\expandafter
            \MT@char\MT@charxstring\relax\relax\relax\relax\relax
      \fi
    \fi
  \fi
}
% new:
\def\MT@is@uni@comp#1\iffontchar#2\else#3\fi\relax{%
  \ifx\\#2\\\else\edef\MT@char{\iffontchar#2\fi}\fi
}
\makeatother

\ifthenelse{\lengthtest { \paperwidth = 11in}}
    { \geometry{margin=0.4in} }
	{\ifthenelse{ \lengthtest{ \paperwidth = 297mm}}
		{\geometry{top=1cm,left=1cm,right=1cm,bottom=1cm} }
		{\geometry{top=1cm,left=1cm,right=1cm,bottom=1cm} }
	}
\pagestyle{empty}
\makeatletter
\renewcommand{\section}{\@startsection{section}{1}{0mm}%
                                {-1ex plus -.5ex minus -.2ex}%
                                {0.5ex plus .2ex}%x
                                {\sffamily\large}}
\renewcommand{\subsection}{\@startsection{subsection}{2}{0mm}%
                                {-1explus -.5ex minus -.2ex}%
                                {0.5ex plus .2ex}%
                                {\sffamily\normalsize\itshape}}
\renewcommand{\subsubsection}{\@startsection{subsubsection}{3}{0mm}%
                                {-1ex plus -.5ex minus -.2ex}%
                                {1ex plus .2ex}%
                                {\normalfont\small\itshape}}
\makeatother
\setcounter{secnumdepth}{0}
\setlength{\parindent}{0pt}
\setlength{\parskip}{0pt plus 0.5ex}
% -----------------------------------------------------------------------

\usepackage{academicons}
\usepackage{mathtools}
\begin{document}
\footnotesize
%\raggedright

\begin{center}
  {\huge\sffamily\bfseries MM I}\\
\end{center}
\setlength{\premulticols}{0pt}
\setlength{\postmulticols}{0pt}
\setlength{\multicolsep}{1pt}
\setlength{\columnsep}{1.8em}
\begin{multicols}{3}
\textbf{Reducibles a variable separable}
\vspace{-5pt}
\[y'=f(ax+by+c) \rightarrow \begin{matrix}
  z=ax+by+c \\
  y' = {(z'-a)}/{b}
\end{matrix} \rightarrow z' = bf(z)+a\]

\vspace{-7pt}
\textbf{Homogéneas}
\vspace{-8pt}
\[y'=f(kx,ky)=f(x,y) \ \forall k \rightarrow \begin{matrix}
  z=y/x \\
  y' = z+xz' \end{matrix}  \rightarrow \mbox{Separable}\]

\vspace{-8pt}
\textbf{Reducibles a homogénea}
\vspace{-3pt}
\[y'=f\left(\frac{ax+by+c}{a'x+b'y+c'}\right); \ \ \boxed{\mbox{Caso A. Si } c=c'=0 \implies \mbox{Homogénea}}\]

\fbox{\begin{minipage}{29em}
\vspace{-5pt}
\[\mbox{B. Si } \left|\begin{matrix}
  a & b \\ a' & b'
\end{matrix}\right| \neq 0; \  \left\{\begin{matrix}
  x=\alpha + h \\ y= \beta +k
\end{matrix}\right. \mbox{ : } \frac{ax+by+c}{a'x+b'y+c'} = \frac{a\alpha+b\beta}{a'\alpha+b'\beta} \]\end{minipage}}

\fbox{\begin{minipage}{29em}
\vspace{-5pt}
\[\mbox{C. Si } \left|\begin{matrix}
  a & b \\ a' & b'
\end{matrix}\right| = 0; y'=f\left(\frac{ax+by+c}{d(ax+by)+c'}\right); \begin{matrix}
  z=ax+by \\ \mbox{Sep.}
\end{matrix}\]
\end{minipage}}

\textbf{Exactas}
\vspace{-6pt}
\[M(x,y)dx+N(x,y)dy=0 \mbox{ es exacta } \iff M=\frac{\partial f}{\partial x}; \ N=\frac{\partial f}{\partial y} \iff \]
\vspace{-10pt}
\[\frac{\partial M}{\partial y} = \frac{\partial N}{\partial x}; \mbox{ Entonces encontramos } f \mbox{ y la solución es } f(x,y)=C\]
\textbf{Cuasi-exactas}
Si $M(x,y)dx+N(x,y)dy=0$ no es exacta, se puede encontrar un $\mu(x,y):$
$\mu \cdot \left(M(x,y)dx+N(x,y)dy\right)=0$ exacta.
\[\frac{\partial(\mu M)}{\partial y}=\frac{\partial(\mu N)}{\partial x}\rightarrow \frac{1}{\mu}\left(N\frac{\partial \mu}{\partial x}-M\frac{\partial \mu}{\partial y}\right)=\frac{\partial M}{\partial y}-\frac{\partial N}{\partial x}\]
Solo puede resolverse suponiendo $\mu = \mu(x) \mbox{ o } \mu=\mu(y)$
\[\mu = \mu(x): \frac{\mu'}{\mu}=\left(\frac{\partial M}{\partial y}-\frac{\partial N}{\partial x}\right)/N=g(x) \mbox{ función solo de x.}\]
\vspace{-7pt}
\[\mu = \mu(y): \frac{\mu'}{\mu}=\left(\frac{\partial N}{\partial x}-\frac{\partial M}{\partial y}\right)/M=h(y) \mbox{ función solo de y.}\]
\vspace{-10pt}

\textbf{Lineales}
\vspace{-8pt}
\[y'+p(x)y=q(x) \mbox{ o } \frac{dx}{dy} + p(y)x=g(y) \ \ \boxed{\mbox{Caso A. } q(x)=0 \implies \mbox{ Sep.}}\]
\vspace{-5pt}
  \[\mbox{Caso B. Si } q(x) \neq 0; \mbox{ Resolvemos } y_h'+p(x)y_h=0 \rightarrow y=c(x)y_h(x)\]
  \[ c'y_h = q(x) \implies \mbox{ Sep.}\]
  \vspace{-18pt}

  \textbf{Bernoulli}
  \vspace{-5pt}
\[y'+p(x)y=q(x)y^n, \ \ z = 1/y^{n-1}; \ \ \frac{-1}{n-1} z' + zp(x)=q(x)\]
\vspace{-9pt}

\textbf{Paramétricas}
\vspace{-3pt}
  \[\mbox{Caso A. } f(y,y')=0 : y=g(y') \rightarrow y'=p; \ \boxed{dx =  g'(p) dp/p}\]
\vspace{-7pt}
  \[\mbox{Caso B. } f(x,y')=0 : x=g(y') \rightarrow y'=p;  \ \boxed{dy = p g'(p)dp}\]
  \vspace{-10pt}

  \textbf{Lagrange}
\vspace{-10pt}
\[y=xf(y')+g(y') \rightarrow y'=p; \ \ \boxed{(f(p)-p) \frac{dx}{dp} +xf'(p)=-g'(p)}\]
\vspace{-10pt}

\textbf{Clairut}
\vspace{-8pt}
\[y=xy'+g(y') \mbox{ (Lagrange) }  \rightarrow dp [x+g'(p)]=0\]
\[x+g'(p) = 0 \mbox{ (SG)}; \ \ dp = 0 \mbox{ (SP)}\]
\vspace{-19pt}

\textbf{Ricatti}
\vspace{-5pt}
\[y' + a(x)y^2+b(x)y+c=0 \ \& \mbox{ SP } y_1 \rightarrow y = y_1 +z; \ \ y' = y_1'+z' \]
\[\mbox{Sustimos y SP = 0} \rightarrow \boxed{z'+(2a y_1 +b)z +az^2=0} \mbox{ (Bernoulli)}\]
\[y' + a(x)y^2+b(x)y+c=0 \ \& \mbox{ 2 SP } y_1; \ y_2 \rightarrow \boxed{\frac{y-y_1}{y-y_2}=Ce^{\int{a(y_2-y_1)dx}}}\]
\textbf{Familias de curvas}
\vspace{-9pt}
\[h(x,y,\{a_i\})=0 \rightarrow \frac{d^i h}{dx^i} =0 \rightarrow \mbox{ quitamos } {a^i};  F\left(x,y,\{y^{(i)}\}\right)=0\]
\vspace{-10pt}
\[F\left(x,y,-\frac{1}{y}\right) \mbox{ es la trayectoria ortogonal a la familia de curvas}\]
\textbf{Reducción de orden}
\[F(x,y^{(k)},\dots,y^{(n)})=0\rightarrow u=y^{(k)} \rightarrow F(x,u,\dots,u^{(n-k)})=0\]
\[F(y,\{y^{(i)}\})=0 \rightarrow y'=p(y); \ y''=p'p; \  \dots \rightarrow F(y,p,\{p^{(j)}\})=0\]
\[F(z,ty,\{ty^{(i)}\})=t^k F(z,y,\{y^{(i)}\}) \ \ \begin{matrix}
  y=e^{\int{z dx}} \\ y'=z e^{\int{z dx}}
\end{matrix} \ , ... F(x,z,\{z^{(j)}\})\]
\textbf{Teorema de Picard.}Si $f(x,y)$ y $\partial f / \partial y$ son funciones continuas sobre un cerrado $\mathbf{R}$ entonces por cada punto $(x_0,y_0)$ del interior de $\mathbf{R}$ pasa una única curva integral de la ecuación $dy/dx = f(x,y)$.

\textbf{Teorema de exist. y unicidad 2º orden lineal.} Sean $P(x)$, $Q(x)$, $R(x)$ continuas en un intervalo cerrado $[a,b]$, entonces si $x_0 \in [a, b]$ e $y_0$ , $y_0'$ son arbitrarios, la EDO: $y'' +P(x)y' +Q(x)y = R(x)$ tiene una única solución $y(x)$ en $[a, b]$ tal que $y(x_0) = y_0$ y $y'(x_0) = y_0'$.

\textbf{SG 2º orden lineal homogénea}
$y_1$ e $y_2$ SPH lin. indp.
\vspace{-5pt}
\[y(x)=C_1 y_1(x)+C_2 y_2(x) \mbox{ SG}; W=y_1 y_2' - y_2 y_1'=K e^{-\int{P dx}} \neq 0\]
\vspace{-15pt}
\[\mbox{Conocida } y_1 \mbox{ SPH }; y_2=u(x)y_1 \mbox{ también SPH}; \ u=\int{\frac{e^{-\int{P dx}}}{y_1^2}dx}\]
\vspace{-10pt}

\textbf{SG 2º orden lineal homogénea Coef. Cte.}
\vspace{-5pt}
\[y''+py'+qy=0; \ \ y(x)=e^{mx}; \ \ m^2+pm+q=0\]
\[y = C_1 e^{m_1x}+C_2 e^{m_2x}; \ y = e^{ax}(C_1 \cos{bx}+C_2\sin{bx})\]
\[y=C_1 e^{-px/2}+C_2 x e^{-px/2} \mbox{ (R \neq; a}\pm\mbox{bi; R =)}\]
\textbf{Ecuación equidimensional de Euler}
\vspace{-5pt}
\[x^2y''+pxy'+qy=0; \ x=e^z \mbox{(coef. cte.) o } y=x^{m}\]
\vspace{-12pt}

\textbf{Transformar EDOH a coef. cte}
\vspace{-4pt}
\[\frac{Q'+2PQ}{Q^{3/2}}=\alpha \implies z=\int{\sqrt{Q}dx},  y''(z)+ky'(z)+cy(z)=0\]
\vspace{-8pt}

\textbf{Lineal inhomogénea de 2 orden}
\vspace{-5pt}
\[y(x)=y_h(x)+y_p(x); \ \ y_h \mbox{ SGH y } y_p \mbox{ SPI}\]
\vspace{-13pt}

\textbf{Coeficientes indeterminados}
\vspace{-5pt}
\[y''+py'+qy=R(x); \ R(x) :  e^{ax} \mbox{ (a)}; \sin{ax} \mbox{ o } \cos{ax}\mbox{ (b)}; \sum_{i=0}^n a_i x^i\mbox{ (c)}\]
\vspace{-8pt}
\[\mbox{(a) } y_p=Ae^{ax}; a^2+pa+q\neq 0 \ \vert \  y_p=Axe^{ax}; 2a+P \neq 0 \ \vert \ y_p=Ax^2e^{ax}\]
\[\mbox{(b) } y_p=A\cos{ax}+B\sin{ax}; (b-q)^2+(bp)^2\neq 0 \ \vert \  \bar{y}_p=x y_p\]
\vspace{-10pt}
\[\mbox{(b) } y_p=\sum_{i=0}^n A_i x^i; q\neq 0 \ \vert \ y_p=x\sum_{i=0}^n A_i x^i\]
\vspace{-6.5pt}

\textbf{Variación de ctes. (Cualquier EDOI 2º)}
\vspace{-5pt}
\[y_p=u_1(x)y_1(x)+u_2(x)y_2(x) \mbox{ a partir de la EDOH}\]
\[u_1'y_1+u_2'y_2=0; \ u_1'y_1'+u_2'y_2'=R(x); \mbox{Resolver sist. lineal}\]

\textbf{Series de Potencias}
\vspace{-5pt}
\[\sum_{n=0}^\infty{a_{n}x^{n}}; R=\lim_{n \to \infty}{\left| \frac{a_{n}}{a_{n+1}} \right|}; \ f(x)\cdot g(x)=\sum_{n=0}^\infty \sum_{i=0}^n{p_{i}q_{n-i}}x^{n} \]
\vspace{-7.5pt}

\textbf{Teorema}: Sea $x_{0}$ un punto ordinario de la EDO $y''$ + $y''+P(x)y'+Q(x)y=0$ y sean $a_{0}$, $a_{1}$ constantes arbitrarias $\implies$ existe una única función $y(x)$ analı́tica en $x_{0}$ que es solución de la EDO en un mismo entorno de $x_{0}$ que $f(x)$ y $g(x)$ y satisface $y(x0) = a0$ ; $y'(x_{0} ) = a_{1}$.
\phantom{ }

\phantom{ }

\phantom{ }

\phantom{ }

\phantom{ }

\phantom{ }

\phantom{ }

\phantom{ }

\phantom{ }

\phantom{ }

\phantom{ }

\phantom{ }

\phantom{ }

\phantom{ }

\phantom{ }

\phantom{ }

\phantom{ }

\phantom{ }

\phantom{ }

\phantom{ }

\phantom{ }

\phantom{ }

\phantom{ }

\phantom{ }

\phantom{ }

\phantom{ }

\phantom{ }

\phantom{ }

\phantom{ }

\phantom{ }

\phantom{ }

\phantom{ }

\phantom{ }

\phantom{ }

\phantom{ }

\phantom{ }

\phantom{ }

\phantom{ }

\phantom{ }

\phantom{ }

\phantom{ }

\phantom{ }

\phantom{ }

\phantom{ }

\phantom{ }

\phantom{ }

\phantom{ }

\phantom{ }

\phantom{ }

\phantom{ }

\phantom{ }

\phantom{ }

\phantom{ }

\phantom{ }

\phantom{ }

\phantom{ }

\phantom{ }

\phantom{ }

\phantom{ }

\phantom{ }

\phantom{ }

\phantom{ }

\phantom{ }

\phantom{ }

\phantom{ }

\phantom{ }

\phantom{ }

\phantom{ }

\phantom{ }

\phantom{ }

\phantom{ }

\phantom{ }

\phantom{ }

\phantom{ }

\phantom{ }

\phantom{ }

\phantom{ }

\phantom{ }

\phantom{ }

\phantom{ }

\phantom{ }

\phantom{ }

\phantom{ }

\phantom{ }

\phantom{ }

\phantom{ }

\phantom{ }

\phantom{ }

\phantom{ }
\end{multicols}

\end{document}
